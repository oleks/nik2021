\begin{abstract}

Contemporary revision control systems (e.g., git) employ line-by-line
differences between files (so-called ``diffs``), to precisely denote
the difference between file revisions. Such diffs are quick to apply
and revert, are often comprehensible to humans, while allowing the
revision control system to function as an incremental backup
system---storing just the differences between revisions.

However, line-by-line ``diffs`` can only be effectively performed on
text-based files. They are ill-suited for other file types, such as
image, video, and PDF files. The design of these file formats often
did not factor in that multiple revisions of such files may need to be
maintained. Yet, in the modern highly digital society, it is becoming
increasingly more important to maintain revisions of arbitrary file
types.

The forward difference between one revision and the next can often
just as precisely be described by an edit command, and a denotation of
the environment in which that command has to be executed. This paper
describes a revision control system that employs such an approach.

Revision control with edit commands, rather than diffs, does suffer
from reduced ability to immediately revert changes, and may increase
the space required to store all the revisions. However, the benefits
it brings may outweigh these downsides in various domains.

\end{abstract}
