\begin{abstract}

Contemporary revision control systems (e.g., git) use so-called
``diffs'' as a basic building block. Under the assumption that a
digital artefact is a sequence of comparable blocks (e.g., a file is a
sequence of lines) --- a classical ``diff'' is a denotation of how two
such sequences differ. We know how to compute such diffs efficiently.
However, user-intervention to fine-pick which changes go into which
revision, usually helps ensure a more auditable trail of revisions.

With humans in the loop, diffs should be human-editable.  This is
conceivable for line-oriented, text-based files (e.g., source code),
but less so for binary files (e.g., PDF). It remains an open problem
how to efficiently control the revision of the latter kind of files,
while ensuring auditability. This is problematic in an increasingly
digital society.

% edit commands and diffs
% edit commands vs. diffs
% edit commands as diffs

In this paper, I expand upon the notion of a diff, to enable a
sequence of edit commands to act as a diff. This allows the user to
reason about a diff at a level of abstraction close to, or equivalent
to the one at which the changes were made.

\end{abstract}
